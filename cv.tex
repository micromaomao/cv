\documentclass[a4paper]{article}
\usepackage[utf8x]{inputenc}
\usepackage[english]{babel}
\usepackage[calc,showdow,english]{datetime2}
\usepackage[margin=1.1cm]{geometry}
\usepackage{helvet}
\renewcommand{\familydefault}{\sfdefault}
\usepackage{tikz}
\usepackage{graphicx}
\usepackage{enumitem}
\usepackage{xcolor}
\usepackage{textcomp}
\usetikzlibrary{calc}
\graphicspath{{icons/}}
\def\newicon#1{%
  \tikz[baseline=0.1em]{
    \path[use as bounding box] (0, 0) rectangle (1em, 1em);
    #1
  }%
}
\def\newsvgicon#1#2{
  \def\svgwidth{1em}%
  \tikz[baseline=-#2]{\node[inner sep=0em, use as bounding box, text width=1em]{\input{#1}}}%
}

\def\uclicon{\newsvgicon{icons/ucl.pdf_tex}{0.95em}}
\def\githubicon{\newsvgicon{icons/github.pdf_tex}{0.85em}}
\def\blueghicon{\hskip 0em\newsvgicon{icons/bluegithub.pdf_tex}{0.3em}}
\def\tflicon{\newsvgicon{icons/tfl.pdf_tex}{0.3em}\hskip -0.15em}
\def\trademark{™}
\def\pencil{\newsvgicon{icons/pencil.pdf_tex}{0.85em}\hskip 0.2em}

\documentclass[a4paper]{article}
\usepackage[utf8x]{inputenc}
\usepackage{microtype}
\usepackage[english]{babel}
\usepackage[calc,showdow,english]{datetime2}
\usepackage[margin=0.8cm]{geometry}
\usepackage[T1]{fontenc}
\usepackage{helvet}
\renewcommand{\familydefault}{\sfdefault}
\usepackage{tikz}
\usepackage{graphicx}
\usepackage{enumitem}
\usepackage{xcolor}
\usepackage{textcomp}
\usepackage{amsmath}
\usepackage{mathtools}
\usepackage{transparent}
\usepackage{setspace}
\usetikzlibrary{calc}
\graphicspath{{icons/}}
\def\newicon#1{%
  \tikz[baseline=0.1em]{
    \path[use as bounding box] (0, 0) rectangle (1em, 1em);
    #1
  }%
}
\def\newsvgicon#1#2{
  \def\svgwidth{1em}%
  \tikz[baseline=-#2]{\node[inner sep=0em, use as bounding box, text width=1em]{\input{#1}}}%
}

\def\uclicon{\newsvgicon{icons/ucl.pdf_tex}{0.95em}}
\def\githubicon{\newsvgicon{icons/github.pdf_tex}{0.85em}}
\def\blueghicon{\hskip 0em\newsvgicon{icons/bluegithub.pdf_tex}{0.3em}}
\def\tflicon{\newsvgicon{icons/tfl.pdf_tex}{0.3em}\hskip -0.15em}
\def\trademark{™}
\def\pencil{\newsvgicon{icons/pencil.pdf_tex}{0.85em}\hskip 0.2em}


\makeatletter
\DeclareRobustCommand\tlabel{%
  \tikz[overlay, remember picture, baseline=(nlabel.base)]{\node[anchor=base, inner sep=0pt] (nlabel) {};}%
}
\DeclareRobustCommand\posttlabel{%
  \tikz[overlay, remember picture]{
    \coordinate (ll) at (nlabel.south -| current page.north west);
    \path[fill=link] ($(ll)-(0.1cm,0pt)$) rectangle ++(0.2cm, 0.9em);
  }%
}
\renewcommand\section[1]{%
  \@startsection{section}{1}{\z@}{0.1\baselineskip}{0.1\baselineskip \@minus 0.1\baselineskip}{\bfseries\MakeUppercase}*{\tlabel{}#1\makebox[0pt]{\posttlabel}}%
}
\makeatother

\def\dashdiv{{%
  \def\sp{0.3em minus 0.1em}%
  \hspace{\sp}%
  \color{black!50!white}%
  --%
  \hspace{\sp}%
}}

\setlength{\parindent}{0em}
\setlength{\parskip}{1em plus 0.5em minus 0.5em}
\setstretch{1.05}

\setitemize{topsep=0em, itemsep=0.5\parskip minus 0.2em, partopsep=0em, parsep=0.5\parskip}
\setlist{leftmargin=1.2em}

\colorlet{link}{blue!50!black}

\makeatletter
\DTMnewdatestyle{my}{%
  \renewcommand{\DTMDisplaydate}[4]{\DTMshortmonthname{##2} ##1}%
  \renewcommand{\DTMdisplaydate}{\DTMDisplaydate}%
}
\DTMsetdatestyle{my}
\usepackage[hidelinks, pdftex, pdfauthor={Tingmao Wang}, pdftitle={Tingmao Wang - \today}]{hyperref}
\makeatother

\pagestyle{empty}%

\def\ghurl#1{%
  \href{#1}{({\blueghicon{}})}%
}


\def\header{%
  \DTMsetdatestyle{my}%
  \begin{tikzpicture}[every node/.style={inner ysep=0.1cm, inner xsep=0cm, anchor=north west}]
    \coordinate (topleft) at (0, 0);
    \coordinate (topright) at (\textwidth, 0);

    \node[font=\Large] (name) at (topleft) {Tingmao Wang};
    \node (bio) at (name.south west) {{\color{black!50!white}\emph{(\today)}} Computer Science undergraduate at\uclicon{}UCL};

    \node[anchor=north east, text width=0.4\textwidth, align=flush right, inner ysep=0cm] (contacts) at (topright)
      {m@maowtm.org \\
        {\color{link}\href{https://github.com/micromaomao}{\githubicon{} micromaomao}}%
          \IfFileExists{phone_number.txt}{\\ \input{phone_number.txt}}{}
        };

    \coordinate (bottomleft) at ($(bio.south west)+(0, -0.1cm)$);
    \coordinate (bottomright) at ($(bottomleft)+(\textwidth, 0)$);

    \path[use as bounding box] (topleft) rectangle (bottomright);

    \path[draw] ($(bottomleft)+(-0.2cm,0)$) -- ($(bottomright)+(0.2cm,0)$);
  \end{tikzpicture}%
}

\begin{document}
  \pagestyle{empty}%
  \header

  \section{Education}

  \begin{itemize}
    \def\ongoing{%
      \tikz[baseline=(t.base)]{
        \node[fill=orange!30!white, use as bounding box, inner sep=0.25em, font=\small] (t) {\textsc{1st year}}
      }%
    }
    \item \ongoing{} \textbf{BSc Computer Science} \dashdiv{} University College London \dashdiv{} 2019-2022

      Relevant modules \& coursework (able to provide code upon request):

      \begin{itemize}

        \item Principles of Programming: C \& Haskell

        \begin{itemize}
          \item Built a journey planner for the\tflicon{} London rail network (tube \& others) entirely in C.

          \begin{itemize}
            \item Shows paths with least station, least interchange or paths that avoid certain fare zones.
          \end{itemize}
        \end{itemize}

        \item Engineering Challenges 1: digital circuit, MIPS architecture.

        \begin{itemize}
          \item Built a simplified MIPS computer (8 instructions implemented) in a 2 person team on FPGA by drawing schematics. Self-taught Verilog and applied to the project.
        \end{itemize}

      \end{itemize}

    \item \textbf{A-Level} \dashdiv{} 2017-2019

      Math, Further Math, Physics and Economics \dashdiv{} A*A*A*A*
  \end{itemize}

  \section{Skills}

  \begin{itemize}[itemsep=0.1\parskip]
    \item Languages: Go, Rust, JavaScript, C, Python. Understands object-oriented and functional programming.

    \begin{itemize}
      \item Currently learning C++ and Haskell.
    \end{itemize}

    \item Linux (scripting, programming \& server administration), Git, Docker.

    \item Algorithm \& data structure:

    \begin{itemize}
      \item Binary search tree, hash table, queue, graphs (Pathfinding, MST, etc.), dynamic programming.
    \end{itemize}

    \item Basic cryptography ((a)symmetric encryption, hashing, Merkle tree, etc.) \& security (web \& native).

  \end{itemize}

  \section{Personal Projects}

  \def\ghurl#1{%
    \href{#1}{({\blueghicon{}})}%
  }

  \begin{itemize}

    \item \href{https://status.maowtm.org}{\color{link}\textbf{status.maowtm.org}} \ghurl{https://github.com/micromaomao/serverwatch}: simple server monitoring with web push notification \dashdiv{} Aug-Sept 2019.

    \begin{itemize}
      \item Backend built with Rust, using the \href{https://rocket.rs}{\color{link}Rocket} framework and SQLite database.
      \item Frontend built with \href{https://svelte.dev/}{\color{link}Svelte}.
      \item Successfully notified me of server fault several times.
    \end{itemize}

    \item \href{https://paper.sc}{\color{link}\textbf{paper.sc}} \ghurl{https://github.com/micromaomao/schsrch}: CIE past paper quick finder \& search engine \dashdiv{} 2016-2019

    \begin{itemize}
      \item Average of \texttildelow{}9000 daily searches for the last 365 days; recommended by several of my high school teacher.
      \item Backend built with Node.js, using MongoDB and Elasticsearch.
      \item Frontend built with React, using Webpack for bundling. Acts as a PWA with ServiceWorker.
      \item Made a PDF viewer: using PDF.js for rendering, wrote own input handling (touch screen \& Macbook trackpad pinch-to-zoom, inertia scrolling, etc.).
      \item Parses the document to find matching question numbers and hence create hyperlinks in the PDF viewer between question paper and mark scheme for the same question.
      \item Includes extensive unit tests, and builds on Travis CI.
    \end{itemize}

    \item \textbf{ts-player} \ghurl{https://github.com/micromaomao/ts-player}: a terminal recorder that produces files capable of efficient random access \dashdiv{} Jan 2019

    \begin{itemize}
      \item Written in Go, used on Linux.
      \item Uses \href{https://developers.google.com/protocol-buffers}{\color{link}protobuf} for storage format and \href{https://facebook.github.io/zstd/}{\color{link}zstd} for compression.
      \item Uses the Linux termios API to execute process in monitored PTY, and a Golang binding of \href{http://www.leonerd.org.uk/code/libvterm/}{\color{link}libvterm} to parse terminal escape sequences (to get the color and position of characters right).
    \end{itemize}

    \item \textbf{go-ecbpass} \ghurl{https://github.com/micromaomao/go-ecbpass}: a deterministic, stateless pseudo-random password generator \dashdiv{} Oct 2018

    \begin{itemize}
      \item Uses scrypt to derive passwords for each website based on its domain and the user's master password.
    \end{itemize}

  \end{itemize}

  \section{Other Interests}

  \begin{itemize}[itemsep=0.1\parskip]
    \item Able to do some basic 3D modelling with Blender \href{https://maowtm.org/Artworks/}{\color{link}(some works I've done)}; also interested in animation and graphic design.
  \end{itemize}

\end{document}
